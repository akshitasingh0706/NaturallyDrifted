%% Generated by Sphinx.
\def\sphinxdocclass{report}
\documentclass[letterpaper,10pt,english]{sphinxmanual}
\ifdefined\pdfpxdimen
   \let\sphinxpxdimen\pdfpxdimen\else\newdimen\sphinxpxdimen
\fi \sphinxpxdimen=.75bp\relax
\ifdefined\pdfimageresolution
    \pdfimageresolution= \numexpr \dimexpr1in\relax/\sphinxpxdimen\relax
\fi
%% let collapsible pdf bookmarks panel have high depth per default
\PassOptionsToPackage{bookmarksdepth=5}{hyperref}

\PassOptionsToPackage{warn}{textcomp}
\usepackage[utf8]{inputenc}
\ifdefined\DeclareUnicodeCharacter
% support both utf8 and utf8x syntaxes
  \ifdefined\DeclareUnicodeCharacterAsOptional
    \def\sphinxDUC#1{\DeclareUnicodeCharacter{"#1}}
  \else
    \let\sphinxDUC\DeclareUnicodeCharacter
  \fi
  \sphinxDUC{00A0}{\nobreakspace}
  \sphinxDUC{2500}{\sphinxunichar{2500}}
  \sphinxDUC{2502}{\sphinxunichar{2502}}
  \sphinxDUC{2514}{\sphinxunichar{2514}}
  \sphinxDUC{251C}{\sphinxunichar{251C}}
  \sphinxDUC{2572}{\textbackslash}
\fi
\usepackage{cmap}
\usepackage[T1]{fontenc}
\usepackage{amsmath,amssymb,amstext}
\usepackage{babel}



\usepackage{tgtermes}
\usepackage{tgheros}
\renewcommand{\ttdefault}{txtt}



\usepackage[Bjarne]{fncychap}
\usepackage{sphinx}

\fvset{fontsize=auto}
\usepackage{geometry}


% Include hyperref last.
\usepackage{hyperref}
% Fix anchor placement for figures with captions.
\usepackage{hypcap}% it must be loaded after hyperref.
% Set up styles of URL: it should be placed after hyperref.
\urlstyle{same}

\addto\captionsenglish{\renewcommand{\contentsname}{Contents:}}

\usepackage{sphinxmessages}
\setcounter{tocdepth}{1}



\title{NaturallyDrifted}
\date{Aug 18, 2022}
\release{}
\author{Akshita Singh}
\newcommand{\sphinxlogo}{\vbox{}}
\renewcommand{\releasename}{}
\makeindex
\begin{document}

\ifdefined\shorthandoff
  \ifnum\catcode`\=\string=\active\shorthandoff{=}\fi
  \ifnum\catcode`\"=\active\shorthandoff{"}\fi
\fi

\pagestyle{empty}
\sphinxmaketitle
\pagestyle{plain}
\sphinxtableofcontents
\pagestyle{normal}
\phantomsection\label{\detokenize{index::doc}}


\sphinxstepscope


\chapter{Base Classes}
\label{\detokenize{baseModules/modules:base-classes}}\label{\detokenize{baseModules/modules::doc}}
\sphinxstepscope


\section{base module}
\label{\detokenize{baseModules/base:module-base}}\label{\detokenize{baseModules/base:base-module}}\label{\detokenize{baseModules/base::doc}}\index{module@\spxentry{module}!base@\spxentry{base}}\index{base@\spxentry{base}!module@\spxentry{module}}\index{detectorParent (class in base)@\spxentry{detectorParent}\spxextra{class in base}}

\begin{fulllineitems}
\phantomsection\label{\detokenize{baseModules/base:base.detectorParent}}
\pysigstartsignatures
\pysiglinewithargsret{\sphinxbfcode{\sphinxupquote{class\DUrole{w}{  }}}\sphinxcode{\sphinxupquote{base.}}\sphinxbfcode{\sphinxupquote{detectorParent}}}{\emph{\DUrole{n}{data\_ref}\DUrole{p}{:}\DUrole{w}{  }\DUrole{n}{Optional\DUrole{p}{{[}}Union\DUrole{p}{{[}}ndarray\DUrole{p}{,}\DUrole{w}{  }list\DUrole{p}{{]}}\DUrole{p}{{]}}}\DUrole{w}{  }\DUrole{o}{=}\DUrole{w}{  }\DUrole{default_value}{None}}, \emph{\DUrole{n}{data\_h0}\DUrole{p}{:}\DUrole{w}{  }\DUrole{n}{Optional\DUrole{p}{{[}}Union\DUrole{p}{{[}}ndarray\DUrole{p}{,}\DUrole{w}{  }list\DUrole{p}{{]}}\DUrole{p}{{]}}}\DUrole{w}{  }\DUrole{o}{=}\DUrole{w}{  }\DUrole{default_value}{None}}, \emph{\DUrole{n}{data\_h1}\DUrole{p}{:}\DUrole{w}{  }\DUrole{n}{Optional\DUrole{p}{{[}}Union\DUrole{p}{{[}}ndarray\DUrole{p}{,}\DUrole{w}{  }list\DUrole{p}{{]}}\DUrole{p}{{]}}}\DUrole{w}{  }\DUrole{o}{=}\DUrole{w}{  }\DUrole{default_value}{None}}, \emph{\DUrole{n}{sample\_dict}\DUrole{p}{:}\DUrole{w}{  }\DUrole{n}{Optional\DUrole{p}{{[}}Dict\DUrole{p}{{]}}}\DUrole{w}{  }\DUrole{o}{=}\DUrole{w}{  }\DUrole{default_value}{None}}, \emph{\DUrole{n}{test}\DUrole{p}{:}\DUrole{w}{  }\DUrole{n}{Union\DUrole{p}{{[}}MMD\DUrole{p}{,}\DUrole{w}{  }LSDD\DUrole{p}{{]}}}\DUrole{w}{  }\DUrole{o}{=}\DUrole{w}{  }\DUrole{default_value}{\textquotesingle{}MMD\textquotesingle{}}}, \emph{\DUrole{n}{sample\_size}\DUrole{p}{:}\DUrole{w}{  }\DUrole{n}{int}\DUrole{w}{  }\DUrole{o}{=}\DUrole{w}{  }\DUrole{default_value}{500}}, \emph{\DUrole{n}{windows}\DUrole{p}{:}\DUrole{w}{  }\DUrole{n}{Optional\DUrole{p}{{[}}int\DUrole{p}{{]}}}\DUrole{w}{  }\DUrole{o}{=}\DUrole{w}{  }\DUrole{default_value}{10}}, \emph{\DUrole{n}{drift\_type}\DUrole{p}{:}\DUrole{w}{  }\DUrole{n}{Optional\DUrole{p}{{[}}Union\DUrole{p}{{[}}Sudden\DUrole{p}{,}\DUrole{w}{  }Gradual\DUrole{p}{{]}}\DUrole{p}{{]}}}\DUrole{w}{  }\DUrole{o}{=}\DUrole{w}{  }\DUrole{default_value}{\textquotesingle{}Sudden\textquotesingle{}}}, \emph{\DUrole{n}{SBERT\_model}\DUrole{p}{:}\DUrole{w}{  }\DUrole{n}{str}\DUrole{w}{  }\DUrole{o}{=}\DUrole{w}{  }\DUrole{default_value}{\textquotesingle{}bert\sphinxhyphen{}base\sphinxhyphen{}uncased\textquotesingle{}}}, \emph{\DUrole{n}{embedding\_model}\DUrole{p}{:}\DUrole{w}{  }\DUrole{n}{Union\DUrole{p}{{[}}Doc2Vec\DUrole{p}{,}\DUrole{w}{  }SBERT\DUrole{p}{,}\DUrole{w}{  }USE\DUrole{p}{{]}}}\DUrole{w}{  }\DUrole{o}{=}\DUrole{w}{  }\DUrole{default_value}{\textquotesingle{}Doc2Vec\textquotesingle{}}}, \emph{\DUrole{n}{transformation}\DUrole{p}{:}\DUrole{w}{  }\DUrole{n}{Optional\DUrole{p}{{[}}Union\DUrole{p}{{[}}UMAP\DUrole{p}{,}\DUrole{w}{  }UAE\DUrole{p}{{]}}\DUrole{p}{{]}}}\DUrole{w}{  }\DUrole{o}{=}\DUrole{w}{  }\DUrole{default_value}{None}}, \emph{\DUrole{n}{pval\_thresh}\DUrole{p}{:}\DUrole{w}{  }\DUrole{n}{int}\DUrole{w}{  }\DUrole{o}{=}\DUrole{w}{  }\DUrole{default_value}{0.05}}, \emph{\DUrole{n}{dist\_thresh}\DUrole{p}{:}\DUrole{w}{  }\DUrole{n}{int}\DUrole{w}{  }\DUrole{o}{=}\DUrole{w}{  }\DUrole{default_value}{0.0009}}, \emph{\DUrole{n}{iterations}\DUrole{p}{:}\DUrole{w}{  }\DUrole{n}{int}\DUrole{w}{  }\DUrole{o}{=}\DUrole{w}{  }\DUrole{default_value}{5}}, \emph{\DUrole{n}{plot}\DUrole{p}{:}\DUrole{w}{  }\DUrole{n}{bool}\DUrole{w}{  }\DUrole{o}{=}\DUrole{w}{  }\DUrole{default_value}{True}}}{}
\pysigstopsignatures
\sphinxAtStartPar
Bases: \sphinxcode{\sphinxupquote{object}}
\index{\_\_init\_\_() (base.detectorParent method)@\spxentry{\_\_init\_\_()}\spxextra{base.detectorParent method}}

\begin{fulllineitems}
\phantomsection\label{\detokenize{baseModules/base:base.detectorParent.__init__}}
\pysigstartsignatures
\pysiglinewithargsret{\sphinxbfcode{\sphinxupquote{\_\_init\_\_}}}{\emph{\DUrole{n}{data\_ref}\DUrole{p}{:}\DUrole{w}{  }\DUrole{n}{Optional\DUrole{p}{{[}}Union\DUrole{p}{{[}}ndarray\DUrole{p}{,}\DUrole{w}{  }list\DUrole{p}{{]}}\DUrole{p}{{]}}}\DUrole{w}{  }\DUrole{o}{=}\DUrole{w}{  }\DUrole{default_value}{None}}, \emph{\DUrole{n}{data\_h0}\DUrole{p}{:}\DUrole{w}{  }\DUrole{n}{Optional\DUrole{p}{{[}}Union\DUrole{p}{{[}}ndarray\DUrole{p}{,}\DUrole{w}{  }list\DUrole{p}{{]}}\DUrole{p}{{]}}}\DUrole{w}{  }\DUrole{o}{=}\DUrole{w}{  }\DUrole{default_value}{None}}, \emph{\DUrole{n}{data\_h1}\DUrole{p}{:}\DUrole{w}{  }\DUrole{n}{Optional\DUrole{p}{{[}}Union\DUrole{p}{{[}}ndarray\DUrole{p}{,}\DUrole{w}{  }list\DUrole{p}{{]}}\DUrole{p}{{]}}}\DUrole{w}{  }\DUrole{o}{=}\DUrole{w}{  }\DUrole{default_value}{None}}, \emph{\DUrole{n}{sample\_dict}\DUrole{p}{:}\DUrole{w}{  }\DUrole{n}{Optional\DUrole{p}{{[}}Dict\DUrole{p}{{]}}}\DUrole{w}{  }\DUrole{o}{=}\DUrole{w}{  }\DUrole{default_value}{None}}, \emph{\DUrole{n}{test}\DUrole{p}{:}\DUrole{w}{  }\DUrole{n}{Union\DUrole{p}{{[}}MMD\DUrole{p}{,}\DUrole{w}{  }LSDD\DUrole{p}{{]}}}\DUrole{w}{  }\DUrole{o}{=}\DUrole{w}{  }\DUrole{default_value}{\textquotesingle{}MMD\textquotesingle{}}}, \emph{\DUrole{n}{sample\_size}\DUrole{p}{:}\DUrole{w}{  }\DUrole{n}{int}\DUrole{w}{  }\DUrole{o}{=}\DUrole{w}{  }\DUrole{default_value}{500}}, \emph{\DUrole{n}{windows}\DUrole{p}{:}\DUrole{w}{  }\DUrole{n}{Optional\DUrole{p}{{[}}int\DUrole{p}{{]}}}\DUrole{w}{  }\DUrole{o}{=}\DUrole{w}{  }\DUrole{default_value}{10}}, \emph{\DUrole{n}{drift\_type}\DUrole{p}{:}\DUrole{w}{  }\DUrole{n}{Optional\DUrole{p}{{[}}Union\DUrole{p}{{[}}Sudden\DUrole{p}{,}\DUrole{w}{  }Gradual\DUrole{p}{{]}}\DUrole{p}{{]}}}\DUrole{w}{  }\DUrole{o}{=}\DUrole{w}{  }\DUrole{default_value}{\textquotesingle{}Sudden\textquotesingle{}}}, \emph{\DUrole{n}{SBERT\_model}\DUrole{p}{:}\DUrole{w}{  }\DUrole{n}{str}\DUrole{w}{  }\DUrole{o}{=}\DUrole{w}{  }\DUrole{default_value}{\textquotesingle{}bert\sphinxhyphen{}base\sphinxhyphen{}uncased\textquotesingle{}}}, \emph{\DUrole{n}{embedding\_model}\DUrole{p}{:}\DUrole{w}{  }\DUrole{n}{Union\DUrole{p}{{[}}Doc2Vec\DUrole{p}{,}\DUrole{w}{  }SBERT\DUrole{p}{,}\DUrole{w}{  }USE\DUrole{p}{{]}}}\DUrole{w}{  }\DUrole{o}{=}\DUrole{w}{  }\DUrole{default_value}{\textquotesingle{}Doc2Vec\textquotesingle{}}}, \emph{\DUrole{n}{transformation}\DUrole{p}{:}\DUrole{w}{  }\DUrole{n}{Optional\DUrole{p}{{[}}Union\DUrole{p}{{[}}UMAP\DUrole{p}{,}\DUrole{w}{  }UAE\DUrole{p}{{]}}\DUrole{p}{{]}}}\DUrole{w}{  }\DUrole{o}{=}\DUrole{w}{  }\DUrole{default_value}{None}}, \emph{\DUrole{n}{pval\_thresh}\DUrole{p}{:}\DUrole{w}{  }\DUrole{n}{int}\DUrole{w}{  }\DUrole{o}{=}\DUrole{w}{  }\DUrole{default_value}{0.05}}, \emph{\DUrole{n}{dist\_thresh}\DUrole{p}{:}\DUrole{w}{  }\DUrole{n}{int}\DUrole{w}{  }\DUrole{o}{=}\DUrole{w}{  }\DUrole{default_value}{0.0009}}, \emph{\DUrole{n}{iterations}\DUrole{p}{:}\DUrole{w}{  }\DUrole{n}{int}\DUrole{w}{  }\DUrole{o}{=}\DUrole{w}{  }\DUrole{default_value}{5}}, \emph{\DUrole{n}{plot}\DUrole{p}{:}\DUrole{w}{  }\DUrole{n}{bool}\DUrole{w}{  }\DUrole{o}{=}\DUrole{w}{  }\DUrole{default_value}{True}}}{}
\pysigstopsignatures
\sphinxAtStartPar
In this class, we define the base arguments and parameters that are required by Alibi
detectors. Not all of these parameters are used by each detector.
\begin{quote}\begin{description}
\sphinxlineitem{Parameters}\begin{itemize}
\item {} 
\sphinxAtStartPar
\sphinxstyleliteralstrong{\sphinxupquote{data\_ref}} (\sphinxstyleliteralemphasis{\sphinxupquote{np.ndarray}}\sphinxstyleliteralemphasis{\sphinxupquote{, }}\sphinxstyleliteralemphasis{\sphinxupquote{list}}) \textendash{} Dataset on which the original model is trained (ex: training dataset). We flag a drift
with a reference to the distribution of this dataset.

\item {} 
\sphinxAtStartPar
\sphinxstyleliteralstrong{\sphinxupquote{data\_h0}} (\sphinxstyleliteralemphasis{\sphinxupquote{np.ndarray}}\sphinxstyleliteralemphasis{\sphinxupquote{, }}\sphinxstyleliteralemphasis{\sphinxupquote{list}}) \textendash{} This is an optional dataset that we can use as a sanity check for the efficacy of a drift
detector. Generally, we use the same dataset as data\_ref (or a stream that comes soon after).
The lack of drift in data\_h0 (with data\_ref as our reference) is the necessary
condition to decide the robustness of the drift detection method

\item {} 
\sphinxAtStartPar
\sphinxstyleliteralstrong{\sphinxupquote{data\_h1}} (\sphinxstyleliteralemphasis{\sphinxupquote{np.ndarray}}\sphinxstyleliteralemphasis{\sphinxupquote{, }}\sphinxstyleliteralemphasis{\sphinxupquote{list}}) \textendash{} This is the principal dataset on which we might see a drift (ex. deployment data).
It can be just one sample (for sudden drifts) or stream of samples (for gradual drifts).
Often, for pipelines, datasets come in batches, and each new batch can then be updated
to the new data\_h1.

\item {} 
\sphinxAtStartPar
\sphinxstyleliteralstrong{\sphinxupquote{sample\_dict}} (\sphinxstyleliteralemphasis{\sphinxupquote{dict}}) \textendash{} Dictionary with samples for reference and comparison data (or streams of comparison data).
The user can directly input the dictionary as our dataset source if they would prefer to
organize the data on their own.

\item {} 
\sphinxAtStartPar
\sphinxstyleliteralstrong{\sphinxupquote{sample\_size}} (\sphinxstyleliteralemphasis{\sphinxupquote{int}}) \textendash{} This parameter decides the number of samples from each of the above 3 datasets that we would
like to work with. For instance, if the entire training data is 100K sentences, we can use
a sample\_size = 500 to randomly sample 500 of those sentences.

\item {} 
\sphinxAtStartPar
\sphinxstyleliteralstrong{\sphinxupquote{test}} (\sphinxstyleliteralemphasis{\sphinxupquote{str}}) \textendash{} Here, we specify the kind of drift detection test we want (KS, KLD, JSD, MMD, LSDD).
Each of them is described in greater detail in the README.md.

\item {} 
\sphinxAtStartPar
\sphinxstyleliteralstrong{\sphinxupquote{drift\_type}} (\sphinxstyleliteralemphasis{\sphinxupquote{str}}) \textendash{} Drifts can vary depending on the time horizan and frequency at which we try to detect
them. This parameter asks the user to specify the type of drift (“Sudden”, “Gradual”,
“Online”). The details of each are in README.md

\item {} 
\sphinxAtStartPar
\sphinxstyleliteralstrong{\sphinxupquote{plot}} (\sphinxstyleliteralemphasis{\sphinxupquote{bool}}) \textendash{} This parameter asks the user if they wish to see some of the plots of the results
from the drift detection. Not every detector will result in relevant plot.

\item {} 
\sphinxAtStartPar
\sphinxstyleliteralstrong{\sphinxupquote{windows}} (\sphinxstyleliteralemphasis{\sphinxupquote{int}}) \textendash{} This parameter is relevant for gradual drifts and helps break down the data into a
certain number of buckets. These buckets can act like “batches” or “data streams”.
The idea behind this approach is that we are trying to localize drifts to a certain
time frame and check for consistencies (or lack thereof) in detection.
If data\_h1 has 100K data points, and if we wish to detect drifts
gradually over time, a proxy approach would be to break the data in sets of 5K points
and then randomly sample from each set separately.

\item {} 
\sphinxAtStartPar
\sphinxstyleliteralstrong{\sphinxupquote{SBERT\_model}} (\sphinxstyleliteralemphasis{\sphinxupquote{str}}) \textendash{} This parameter is specific to the SBERT embedding models. If we choose to work with SBERT,
we can specify the type of SBERT embedding out here. Ex. ‘bert\sphinxhyphen{}base\sphinxhyphen{}uncased’

\item {} 
\sphinxAtStartPar
\sphinxstyleliteralstrong{\sphinxupquote{ert}} (\sphinxstyleliteralemphasis{\sphinxupquote{int}}\sphinxstyleliteralemphasis{\sphinxupquote{ (}}\sphinxstyleliteralemphasis{\sphinxupquote{optional}}\sphinxstyleliteralemphasis{\sphinxupquote{)}}) \textendash{} Expected Run Time before a drift is detected. Alibi detect uses this approach for it’s
online drift detectors. If the average ERT for the reference data is significantly higher
than the average run time for the drifted data, that might indicate a possible drift.

\item {} 
\sphinxAtStartPar
\sphinxstyleliteralstrong{\sphinxupquote{window\_size}} (\sphinxstyleliteralemphasis{\sphinxupquote{int}}) \textendash{} This parameter is used within Alibi’s online detectors.
It specifies the number of datapoints to include in one window.

\item {} 
\sphinxAtStartPar
\sphinxstyleliteralstrong{\sphinxupquote{n\_run}} (\sphinxstyleliteralemphasis{\sphinxupquote{int}}) \textendash{} This parameter is used within Alibi’s online detectors and specifies the number of runs
the detector must perform before we can get an average ERT.

\item {} 
\sphinxAtStartPar
\sphinxstyleliteralstrong{\sphinxupquote{n\_bootstraps}} (\sphinxstyleliteralemphasis{\sphinxupquote{int}}) \textendash{} This parameter is used within Alibi’s online detectors

\item {} 
\sphinxAtStartPar
\sphinxstyleliteralstrong{\sphinxupquote{context\_type}} (\sphinxstyleliteralemphasis{\sphinxupquote{str}}) \textendash{} Context that we wish to ignore
1) sub\sphinxhyphen{}population: if we wish to ignore the relative change in sub\sphinxhyphen{}population of certain
classes

\item {} 
\sphinxAtStartPar
\sphinxstyleliteralstrong{\sphinxupquote{embedding\_model}} (\sphinxstyleliteralemphasis{\sphinxupquote{str}}) \textendash{} This is the principle parameter of this class. It decided the kind of embedding the text
goes through. The embeddings we consider thus far are:
a) SBERT: A Python framework for state\sphinxhyphen{}of\sphinxhyphen{}the\sphinxhyphen{}art sentence, text and image embeddings.
b) Universal Sentence Encoders: USE encodes text into high dimensional vectors that can be
used for text classification, semantic similarity, clustering, and other natural language tasks
c) Doc2Vec: a generalization of Word2Vec, which in turn is an algorithm that uses a
neural network model to learn word associations from a large corpus of text

\end{itemize}

\sphinxlineitem{Return type}
\sphinxAtStartPar
Nothing

\end{description}\end{quote}

\end{fulllineitems}


\end{fulllineitems}


\sphinxstepscope


\section{sampling module}
\label{\detokenize{baseModules/sampling:module-sampling}}\label{\detokenize{baseModules/sampling:sampling-module}}\label{\detokenize{baseModules/sampling::doc}}\index{module@\spxentry{module}!sampling@\spxentry{sampling}}\index{sampling@\spxentry{sampling}!module@\spxentry{module}}\index{samplingData (class in sampling)@\spxentry{samplingData}\spxextra{class in sampling}}

\begin{fulllineitems}
\phantomsection\label{\detokenize{baseModules/sampling:sampling.samplingData}}
\pysigstartsignatures
\pysiglinewithargsret{\sphinxbfcode{\sphinxupquote{class\DUrole{w}{  }}}\sphinxcode{\sphinxupquote{sampling.}}\sphinxbfcode{\sphinxupquote{samplingData}}}{\emph{\DUrole{o}{*}\DUrole{n}{args}}, \emph{\DUrole{o}{**}\DUrole{n}{kwargs}}}{}
\pysigstopsignatures
\sphinxAtStartPar
Bases: {\hyperref[\detokenize{baseModules/base:base.detectorParent}]{\sphinxcrossref{\sphinxcode{\sphinxupquote{detectorParent}}}}}
\index{\_\_init\_\_() (sampling.samplingData method)@\spxentry{\_\_init\_\_()}\spxextra{sampling.samplingData method}}

\begin{fulllineitems}
\phantomsection\label{\detokenize{baseModules/sampling:sampling.samplingData.__init__}}
\pysigstartsignatures
\pysiglinewithargsret{\sphinxbfcode{\sphinxupquote{\_\_init\_\_}}}{\emph{\DUrole{o}{*}\DUrole{n}{args}}, \emph{\DUrole{o}{**}\DUrole{n}{kwargs}}}{}
\pysigstopsignatures
\sphinxAtStartPar
Takes in 2\sphinxhyphen{}3 datasets \sphinxhyphen{} the reference set, the h0 set (generally the same as reference set),
and the possibly drifted set and then samples from them based on the kind of drift we are
trying to flag. The h0 data is optional.
\begin{quote}\begin{description}
\sphinxlineitem{Return type}
\sphinxAtStartPar
Dictionary with dataset (referemce, h0, h1) as key and a numpy array of data samples as values

\end{description}\end{quote}

\end{fulllineitems}

\index{random\_sample() (sampling.samplingData method)@\spxentry{random\_sample()}\spxextra{sampling.samplingData method}}

\begin{fulllineitems}
\phantomsection\label{\detokenize{baseModules/sampling:sampling.samplingData.random_sample}}
\pysigstartsignatures
\pysiglinewithargsret{\sphinxbfcode{\sphinxupquote{random\_sample}}}{\emph{\DUrole{n}{data}\DUrole{p}{:}\DUrole{w}{  }\DUrole{n}{Union\DUrole{p}{{[}}ndarray\DUrole{p}{,}\DUrole{w}{  }list\DUrole{p}{{]}}}}}{}
\pysigstopsignatures
\end{fulllineitems}

\index{sample\_data\_gradual() (sampling.samplingData method)@\spxentry{sample\_data\_gradual()}\spxextra{sampling.samplingData method}}

\begin{fulllineitems}
\phantomsection\label{\detokenize{baseModules/sampling:sampling.samplingData.sample_data_gradual}}
\pysigstartsignatures
\pysiglinewithargsret{\sphinxbfcode{\sphinxupquote{sample\_data\_gradual}}}{}{}
\pysigstopsignatures
\sphinxAtStartPar
Takes in 2\sphinxhyphen{}3 datasets \sphinxhyphen{} the reference set, the h0 set (generally the same as reference set),
and the possibly drifted set and then samples from them based on the kind of drift we are
trying to flag. The h0 data is optional.
\begin{quote}\begin{description}
\sphinxlineitem{Return type}
\sphinxAtStartPar
Dictionary with dataset (referemce, h0, h1) as key and a numpy array of data samples as values

\end{description}\end{quote}

\end{fulllineitems}

\index{sample\_data\_online() (sampling.samplingData method)@\spxentry{sample\_data\_online()}\spxextra{sampling.samplingData method}}

\begin{fulllineitems}
\phantomsection\label{\detokenize{baseModules/sampling:sampling.samplingData.sample_data_online}}
\pysigstartsignatures
\pysiglinewithargsret{\sphinxbfcode{\sphinxupquote{sample\_data\_online}}}{}{}
\pysigstopsignatures
\sphinxAtStartPar
Takes in 2\sphinxhyphen{}3 datasets \sphinxhyphen{} the reference set, the h0 set (generally the same as reference set),
and the possibly drifted set and samples from each given the sample\_size as decided by the
user. The h1 (comparison data) is divded into buckets as decided by the number of windows
\begin{quote}\begin{description}
\sphinxlineitem{Return type}
\sphinxAtStartPar
Dictionary with (0 (referemce), 1 (h0), 2,…, n (h1//window)) and numpy array of data samples as values

\end{description}\end{quote}

\end{fulllineitems}

\index{sample\_data\_sudden() (sampling.samplingData method)@\spxentry{sample\_data\_sudden()}\spxextra{sampling.samplingData method}}

\begin{fulllineitems}
\phantomsection\label{\detokenize{baseModules/sampling:sampling.samplingData.sample_data_sudden}}
\pysigstartsignatures
\pysiglinewithargsret{\sphinxbfcode{\sphinxupquote{sample\_data\_sudden}}}{}{}
\pysigstopsignatures
\sphinxAtStartPar
Takes in 2\sphinxhyphen{}3 datasets \sphinxhyphen{} the reference set, the h0 set (generally the same as reference set),
and the possibly drifted set and samples from each given the sample\_size as decided by the
user. The h0 data is optional.
\begin{quote}\begin{description}
\sphinxlineitem{Return type}
\sphinxAtStartPar
Dictionary with 2\sphinxhyphen{}3 keys (0 (referemce), 1 (h0), 2 (h1)) and numpy array of data samples as values

\end{description}\end{quote}

\end{fulllineitems}

\index{samples() (sampling.samplingData method)@\spxentry{samples()}\spxextra{sampling.samplingData method}}

\begin{fulllineitems}
\phantomsection\label{\detokenize{baseModules/sampling:sampling.samplingData.samples}}
\pysigstartsignatures
\pysiglinewithargsret{\sphinxbfcode{\sphinxupquote{samples}}}{}{}
\pysigstopsignatures
\sphinxAtStartPar
Takes in 2\sphinxhyphen{}3 datasets \sphinxhyphen{} the reference set, the h0 set (generally the same as reference set),
and the possibly drifted set and then samples from them based on the kind of drift we are
trying to flag. The h0 data is optional.
\begin{quote}\begin{description}
\sphinxlineitem{Return type}
\sphinxAtStartPar
Dictionary with dataset (referemce, h0, h1) as key and a numpy array of data samples as values

\end{description}\end{quote}

\end{fulllineitems}


\end{fulllineitems}


\sphinxstepscope


\chapter{Alibi Detectors}
\label{\detokenize{alibiModules/modules:alibi-detectors}}\label{\detokenize{alibiModules/modules::doc}}
\sphinxstepscope


\section{basicDetectors module}
\label{\detokenize{alibiModules/basicDetectors:module-basicDetectors}}\label{\detokenize{alibiModules/basicDetectors:basicdetectors-module}}\label{\detokenize{alibiModules/basicDetectors::doc}}\index{module@\spxentry{module}!basicDetectors@\spxentry{basicDetectors}}\index{basicDetectors@\spxentry{basicDetectors}!module@\spxentry{module}}
\sphinxAtStartPar
sudden and (basic) gradual drifts on text data from the following detectors \sphinxhyphen{} MMD and LSDD
\index{basicDetectors (class in basicDetectors)@\spxentry{basicDetectors}\spxextra{class in basicDetectors}}

\begin{fulllineitems}
\phantomsection\label{\detokenize{alibiModules/basicDetectors:basicDetectors.basicDetectors}}
\pysigstartsignatures
\pysiglinewithargsret{\sphinxbfcode{\sphinxupquote{class\DUrole{w}{  }}}\sphinxcode{\sphinxupquote{basicDetectors.}}\sphinxbfcode{\sphinxupquote{basicDetectors}}}{\emph{\DUrole{o}{*}\DUrole{n}{args}}, \emph{\DUrole{o}{**}\DUrole{n}{kwargs}}}{}
\pysigstopsignatures
\sphinxAtStartPar
Bases: {\hyperref[\detokenize{baseModules/sampling:sampling.samplingData}]{\sphinxcrossref{\sphinxcode{\sphinxupquote{samplingData}}}}}, {\hyperref[\detokenize{baseModules/base:base.detectorParent}]{\sphinxcrossref{\sphinxcode{\sphinxupquote{detectorParent}}}}}
\index{\_\_init\_\_() (basicDetectors.basicDetectors method)@\spxentry{\_\_init\_\_()}\spxextra{basicDetectors.basicDetectors method}}

\begin{fulllineitems}
\phantomsection\label{\detokenize{alibiModules/basicDetectors:basicDetectors.basicDetectors.__init__}}
\pysigstartsignatures
\pysiglinewithargsret{\sphinxbfcode{\sphinxupquote{\_\_init\_\_}}}{\emph{\DUrole{o}{*}\DUrole{n}{args}}, \emph{\DUrole{o}{**}\DUrole{n}{kwargs}}}{}
\pysigstopsignatures
\sphinxAtStartPar
In this class, we check for possible sudden drift in the data, using some of Alibi’s methods.
Sudden drifts are drifts we could see right after deployment.
We can also use sudden drift techniques to try identifying drifts in
a new batch of data (Ex. data being streamed weekly).
\begin{quote}\begin{description}
\sphinxlineitem{Returns}
\sphinxAtStartPar
\begin{itemize}
\item {} 
\sphinxAtStartPar
\sphinxstyleemphasis{Lists and plots of relevant test statistics (p\sphinxhyphen{}values, distances) given the selected}

\item {} 
\sphinxAtStartPar
\sphinxstyleemphasis{detector (MMD, LSDD etc)}

\end{itemize}


\end{description}\end{quote}

\end{fulllineitems}

\index{detector() (basicDetectors.basicDetectors method)@\spxentry{detector()}\spxextra{basicDetectors.basicDetectors method}}

\begin{fulllineitems}
\phantomsection\label{\detokenize{alibiModules/basicDetectors:basicDetectors.basicDetectors.detector}}
\pysigstartsignatures
\pysiglinewithargsret{\sphinxbfcode{\sphinxupquote{detector}}}{}{}
\pysigstopsignatures
\sphinxAtStartPar
Here, we call the relevant drift detection method from Alibi Detect, given user input.
The function uses reference samples and preprocessing from the previous function as arguments
for the detection model development here.
\begin{quote}\begin{description}
\sphinxlineitem{Return type}
\sphinxAtStartPar
A trained detection model (MMD, LSDD etc) as specified by the user input

\end{description}\end{quote}

\end{fulllineitems}

\index{embedData() (basicDetectors.basicDetectors method)@\spxentry{embedData()}\spxextra{basicDetectors.basicDetectors method}}

\begin{fulllineitems}
\phantomsection\label{\detokenize{alibiModules/basicDetectors:basicDetectors.basicDetectors.embedData}}
\pysigstartsignatures
\pysiglinewithargsret{\sphinxbfcode{\sphinxupquote{embedData}}}{}{}
\pysigstopsignatures
\sphinxAtStartPar
Call the samplingData class to construct samples from the input data provided by the user
\begin{quote}\begin{description}
\sphinxlineitem{Return type}
\sphinxAtStartPar
Dictionary with samples for reference and comparison data (or streams of comparison data).

\end{description}\end{quote}

\end{fulllineitems}

\index{preprocess() (basicDetectors.basicDetectors method)@\spxentry{preprocess()}\spxextra{basicDetectors.basicDetectors method}}

\begin{fulllineitems}
\phantomsection\label{\detokenize{alibiModules/basicDetectors:basicDetectors.basicDetectors.preprocess}}
\pysigstartsignatures
\pysiglinewithargsret{\sphinxbfcode{\sphinxupquote{preprocess}}}{}{}
\pysigstopsignatures
\sphinxAtStartPar
Here we process the text data in the following manner:
1) Embed it (generally, by using some kind of a Sentence Transformer)
2) Prepare a dimension reduction model for it that we can than feed into the main Alibi
detector function
\begin{quote}\begin{description}
\sphinxlineitem{Return type}
\sphinxAtStartPar
A dimesnion reduction/preprocessing model that the Alibi Detector can use (generally, an Untrained Autoencoder)

\end{description}\end{quote}

\end{fulllineitems}

\index{run() (basicDetectors.basicDetectors method)@\spxentry{run()}\spxextra{basicDetectors.basicDetectors method}}

\begin{fulllineitems}
\phantomsection\label{\detokenize{alibiModules/basicDetectors:basicDetectors.basicDetectors.run}}
\pysigstartsignatures
\pysiglinewithargsret{\sphinxbfcode{\sphinxupquote{run}}}{}{}
\pysigstopsignatures
\sphinxAtStartPar
Here, we run the detection model from the previous function, on the comparison data on
which we want to check for a possible drift.
\begin{quote}\begin{description}
\sphinxlineitem{Return type}
\sphinxAtStartPar
Lists and plots of relevant test statistics (p\sphinxhyphen{}values, distances) given the selected detector (MMD, LSDD etc)

\end{description}\end{quote}

\end{fulllineitems}

\index{run\_all() (basicDetectors.basicDetectors method)@\spxentry{run\_all()}\spxextra{basicDetectors.basicDetectors method}}

\begin{fulllineitems}
\phantomsection\label{\detokenize{alibiModules/basicDetectors:basicDetectors.basicDetectors.run_all}}
\pysigstartsignatures
\pysiglinewithargsret{\sphinxbfcode{\sphinxupquote{run\_all}}}{}{}
\pysigstopsignatures
\end{fulllineitems}

\index{sampleData() (basicDetectors.basicDetectors method)@\spxentry{sampleData()}\spxextra{basicDetectors.basicDetectors method}}

\begin{fulllineitems}
\phantomsection\label{\detokenize{alibiModules/basicDetectors:basicDetectors.basicDetectors.sampleData}}
\pysigstartsignatures
\pysiglinewithargsret{\sphinxbfcode{\sphinxupquote{sampleData}}}{}{}
\pysigstopsignatures
\sphinxAtStartPar
Call the samplingData class to construct samples from the input data provided by the user
\begin{quote}\begin{description}
\sphinxlineitem{Return type}
\sphinxAtStartPar
Dictionary with samples for reference and comparison data (or streams of comparison data).

\end{description}\end{quote}

\end{fulllineitems}


\end{fulllineitems}


\sphinxstepscope


\section{onlineDetectors module}
\label{\detokenize{alibiModules/onlineDetectors:module-onlineDetectors}}\label{\detokenize{alibiModules/onlineDetectors:onlinedetectors-module}}\label{\detokenize{alibiModules/onlineDetectors::doc}}\index{module@\spxentry{module}!onlineDetectors@\spxentry{onlineDetectors}}\index{onlineDetectors@\spxentry{onlineDetectors}!module@\spxentry{module}}
\sphinxAtStartPar
online (calibrated gradual) drifts on text data from the following detectors \sphinxhyphen{} MMD and LSDD
\index{onlineDetectors (class in onlineDetectors)@\spxentry{onlineDetectors}\spxextra{class in onlineDetectors}}

\begin{fulllineitems}
\phantomsection\label{\detokenize{alibiModules/onlineDetectors:onlineDetectors.onlineDetectors}}
\pysigstartsignatures
\pysiglinewithargsret{\sphinxbfcode{\sphinxupquote{class\DUrole{w}{  }}}\sphinxcode{\sphinxupquote{onlineDetectors.}}\sphinxbfcode{\sphinxupquote{onlineDetectors}}}{\emph{\DUrole{o}{*}\DUrole{n}{args}}, \emph{\DUrole{o}{**}\DUrole{n}{kwargs}}}{}
\pysigstopsignatures
\sphinxAtStartPar
Bases: {\hyperref[\detokenize{baseModules/sampling:sampling.samplingData}]{\sphinxcrossref{\sphinxcode{\sphinxupquote{samplingData}}}}}, {\hyperref[\detokenize{baseModules/base:base.detectorParent}]{\sphinxcrossref{\sphinxcode{\sphinxupquote{detectorParent}}}}}
\index{\_\_init\_\_() (onlineDetectors.onlineDetectors method)@\spxentry{\_\_init\_\_()}\spxextra{onlineDetectors.onlineDetectors method}}

\begin{fulllineitems}
\phantomsection\label{\detokenize{alibiModules/onlineDetectors:onlineDetectors.onlineDetectors.__init__}}
\pysigstartsignatures
\pysiglinewithargsret{\sphinxbfcode{\sphinxupquote{\_\_init\_\_}}}{\emph{\DUrole{o}{*}\DUrole{n}{args}}, \emph{\DUrole{o}{**}\DUrole{n}{kwargs}}}{}
\pysigstopsignatures
\sphinxAtStartPar
Checks for possible drift in the dataset in an online fashion. Instead of detecting drifts
for each new, non\sphinxhyphen{}overlapping window, this method tries to detect drift as soon as any new
data arrives. This detector leverages a calibration method discussed in Cobb et all (2021).
The detectors compute a test statistic  during the configuration phase. Then, at test time,
the test statistic is updated sequentially at a low cost. When no drift has occurred the test
statistic fluctuates around its expected value, and once drift occurs the test statistic starts
to drift upwards. When it exceeds some preconfigured threshold value, drift is detected.

\sphinxAtStartPar
Almost all offline drift detectors have their online counterparts.
\begin{quote}\begin{description}
\sphinxlineitem{Returns}
\sphinxAtStartPar
\begin{itemize}
\item {} 
\sphinxAtStartPar
\sphinxstyleemphasis{1) Lists and plots of expected run times (OnlineMMD, OnlineLSDD etc).}

\item {} 
\sphinxAtStartPar
\sphinxstyleemphasis{2) Plots of  dynamic threshold pitted against the test statistic for that window}

\end{itemize}


\end{description}\end{quote}

\end{fulllineitems}

\index{detector() (onlineDetectors.onlineDetectors method)@\spxentry{detector()}\spxextra{onlineDetectors.onlineDetectors method}}

\begin{fulllineitems}
\phantomsection\label{\detokenize{alibiModules/onlineDetectors:onlineDetectors.onlineDetectors.detector}}
\pysigstartsignatures
\pysiglinewithargsret{\sphinxbfcode{\sphinxupquote{detector}}}{}{}
\pysigstopsignatures
\sphinxAtStartPar
Here, we call the relevant drift detection method from Alibi Detect, given user input.
The function uses reference samples and preprocessing from the previous function as arguments
for the detection model development here.
\begin{quote}\begin{description}
\sphinxlineitem{Return type}
\sphinxAtStartPar
A trained detection model (MMD, LSDD etc) as specified by the user input

\end{description}\end{quote}

\end{fulllineitems}

\index{preprocess() (onlineDetectors.onlineDetectors method)@\spxentry{preprocess()}\spxextra{onlineDetectors.onlineDetectors method}}

\begin{fulllineitems}
\phantomsection\label{\detokenize{alibiModules/onlineDetectors:onlineDetectors.onlineDetectors.preprocess}}
\pysigstartsignatures
\pysiglinewithargsret{\sphinxbfcode{\sphinxupquote{preprocess}}}{}{}
\pysigstopsignatures
\sphinxAtStartPar
Here we process the text data in the following manner:
1) Embed it (generally, by using some kind of a Sentence Transformer)
2) Prepare a dimension reduction model for it that we can than feed into the main Alibi
detector function
\begin{quote}\begin{description}
\sphinxlineitem{Return type}
\sphinxAtStartPar
A dimesnion reduction/preprocessing model that the Alibi Detector can use (generally, an Untrained Autoencoder)

\end{description}\end{quote}

\end{fulllineitems}

\index{run() (onlineDetectors.onlineDetectors method)@\spxentry{run()}\spxextra{onlineDetectors.onlineDetectors method}}

\begin{fulllineitems}
\phantomsection\label{\detokenize{alibiModules/onlineDetectors:onlineDetectors.onlineDetectors.run}}
\pysigstartsignatures
\pysiglinewithargsret{\sphinxbfcode{\sphinxupquote{run}}}{}{}
\pysigstopsignatures
\sphinxAtStartPar
Here, we run the detection model from the previous function, on the comparison data on
which we want to check for a possible drift.
\begin{quote}\begin{description}
\sphinxlineitem{Returns}
\sphinxAtStartPar
\begin{itemize}
\item {} 
\sphinxAtStartPar
\sphinxstyleemphasis{1) Lists and plots of expected run times (OnlineMMD, OnlineLSDD etc).}

\item {} 
\sphinxAtStartPar
\sphinxstyleemphasis{2) Plots of  dynamic threshold pitted against the test statistic for that window}

\end{itemize}


\end{description}\end{quote}

\end{fulllineitems}

\index{sampleData() (onlineDetectors.onlineDetectors method)@\spxentry{sampleData()}\spxextra{onlineDetectors.onlineDetectors method}}

\begin{fulllineitems}
\phantomsection\label{\detokenize{alibiModules/onlineDetectors:onlineDetectors.onlineDetectors.sampleData}}
\pysigstartsignatures
\pysiglinewithargsret{\sphinxbfcode{\sphinxupquote{sampleData}}}{}{}
\pysigstopsignatures
\sphinxAtStartPar
Call the samplingData class to construct samples from the input data provided by the user
\begin{quote}\begin{description}
\sphinxlineitem{Return type}
\sphinxAtStartPar
Dictionary with samples for reference and comparison data (or streams of comparison data).

\end{description}\end{quote}

\end{fulllineitems}


\end{fulllineitems}


\sphinxstepscope


\section{alibiDetectors module}
\label{\detokenize{alibiModules/alibiDetectors:module-alibiDetectors}}\label{\detokenize{alibiModules/alibiDetectors:alibidetectors-module}}\label{\detokenize{alibiModules/alibiDetectors::doc}}\index{module@\spxentry{module}!alibiDetectors@\spxentry{alibiDetectors}}\index{alibiDetectors@\spxentry{alibiDetectors}!module@\spxentry{module}}\index{alibiDetectors (class in alibiDetectors)@\spxentry{alibiDetectors}\spxextra{class in alibiDetectors}}

\begin{fulllineitems}
\phantomsection\label{\detokenize{alibiModules/alibiDetectors:alibiDetectors.alibiDetectors}}
\pysigstartsignatures
\pysiglinewithargsret{\sphinxbfcode{\sphinxupquote{class\DUrole{w}{  }}}\sphinxcode{\sphinxupquote{alibiDetectors.}}\sphinxbfcode{\sphinxupquote{alibiDetectors}}}{\emph{\DUrole{o}{*}\DUrole{n}{args}}, \emph{\DUrole{o}{**}\DUrole{n}{kwargs}}}{}
\pysigstopsignatures
\sphinxAtStartPar
Bases: {\hyperref[\detokenize{baseModules/base:base.detectorParent}]{\sphinxcrossref{\sphinxcode{\sphinxupquote{detectorParent}}}}}
\index{\_\_init\_\_() (alibiDetectors.alibiDetectors method)@\spxentry{\_\_init\_\_()}\spxextra{alibiDetectors.alibiDetectors method}}

\begin{fulllineitems}
\phantomsection\label{\detokenize{alibiModules/alibiDetectors:alibiDetectors.alibiDetectors.__init__}}
\pysigstartsignatures
\pysiglinewithargsret{\sphinxbfcode{\sphinxupquote{\_\_init\_\_}}}{\emph{\DUrole{o}{*}\DUrole{n}{args}}, \emph{\DUrole{o}{**}\DUrole{n}{kwargs}}}{}
\pysigstopsignatures
\sphinxAtStartPar
This is final wrapper class for all text related Alibi Detectors (basic detectors, online
detectors, context aware detectors etc.). We can generally just directly call this one function
and populate all the parameters (datasets, detection tests, drift types etc.) and get our test
statistics.
\begin{quote}\begin{description}
\sphinxlineitem{Parameters}\begin{itemize}
\item {} 
\sphinxAtStartPar
\sphinxstyleliteralstrong{\sphinxupquote{data\_ref}} (\sphinxstyleliteralemphasis{\sphinxupquote{np.ndarray}}\sphinxstyleliteralemphasis{\sphinxupquote{, }}\sphinxstyleliteralemphasis{\sphinxupquote{list}}) \textendash{} Dataset on which the original model is trained (ex: training dataset). We flag a drift
with a reference to the distribution of this dataset.

\item {} 
\sphinxAtStartPar
\sphinxstyleliteralstrong{\sphinxupquote{data\_h0}} (\sphinxstyleliteralemphasis{\sphinxupquote{np.ndarray}}\sphinxstyleliteralemphasis{\sphinxupquote{, }}\sphinxstyleliteralemphasis{\sphinxupquote{list}}) \textendash{} This is an optional dataset that we can use as a sanity check for the efficacy of a drift
detector. Generally, we use the same dataset as data\_ref (or a stream that comes soon after).
The lack of drift in data\_h0 (with data\_ref as our reference) is the necessary
condition to decide the robustness of the drift detection method

\item {} 
\sphinxAtStartPar
\sphinxstyleliteralstrong{\sphinxupquote{data\_h1}} (\sphinxstyleliteralemphasis{\sphinxupquote{np.ndarray}}\sphinxstyleliteralemphasis{\sphinxupquote{, }}\sphinxstyleliteralemphasis{\sphinxupquote{list}}) \textendash{} This is the principal dataset on which we might see a drift (ex. deployment data).
It can be just one sample (for sudden drifts) or stream of samples (for gradual drifts).
Often, for pipelines, datasets come in batches, and each new batch can then be updated
to the new data\_h1.

\item {} 
\sphinxAtStartPar
\sphinxstyleliteralstrong{\sphinxupquote{sample\_dict}} (\sphinxstyleliteralemphasis{\sphinxupquote{dict}}) \textendash{} Dictionary with samples for reference and comparison data (or streams of comparison data).
The user can directly input the dictionary as our dataset source if they would prefer to
organize the data on their own.

\item {} 
\sphinxAtStartPar
\sphinxstyleliteralstrong{\sphinxupquote{sample\_size}} (\sphinxstyleliteralemphasis{\sphinxupquote{int}}) \textendash{} This parameter decides the number of samples from each of the above 3 datasets that we would
like to work with. For instance, if the entire training data is 100K sentences, we can use
a sample\_size = 500 to randomly sample 500 of those sentences.

\item {} 
\sphinxAtStartPar
\sphinxstyleliteralstrong{\sphinxupquote{test}} (\sphinxstyleliteralemphasis{\sphinxupquote{str}}) \textendash{} Here, we specify the kind of drift detection test we want (KS, KLD, JSD, MMD, LSDD).
Each of them is described in greater detail in the README.md.

\item {} 
\sphinxAtStartPar
\sphinxstyleliteralstrong{\sphinxupquote{drift\_type}} (\sphinxstyleliteralemphasis{\sphinxupquote{str}}) \textendash{} Drifts can vary depending on the time horizan and frequency at which we try to detect
them. This parameter asks the user to specify the type of drift (“Sudden”, “Gradual”,
“Online”). The details of each are in README.md

\item {} 
\sphinxAtStartPar
\sphinxstyleliteralstrong{\sphinxupquote{plot}} (\sphinxstyleliteralemphasis{\sphinxupquote{bool}}) \textendash{} This parameter asks the user if they wish to see some of the plots of the results
from the drift detection. Not every detector will result in relevant plot.

\item {} 
\sphinxAtStartPar
\sphinxstyleliteralstrong{\sphinxupquote{windows}} (\sphinxstyleliteralemphasis{\sphinxupquote{int}}) \textendash{} This parameter is relevant for gradual drifts and helps break down the data into a
certain number of buckets. These buckets can act like “batches” or “data streams”.
The idea behind this approach is that we are trying to localize drifts to a certain
time frame and check for consistencies (or lack thereof) in detection.
If data\_h1 has 100K data points, and if we wish to detect drifts
gradually over time, a proxy approach would be to break the data in sets of 5K points
and then randomly sample from each set separately.

\item {} 
\sphinxAtStartPar
\sphinxstyleliteralstrong{\sphinxupquote{SBERT\_model}} (\sphinxstyleliteralemphasis{\sphinxupquote{str}}) \textendash{} This parameter is specific to the SBERT embedding models. If we choose to work with SBERT,
we can specify the type of SBERT embedding out here. Ex. ‘bert\sphinxhyphen{}base\sphinxhyphen{}uncased’

\item {} 
\sphinxAtStartPar
\sphinxstyleliteralstrong{\sphinxupquote{ert}} (\sphinxstyleliteralemphasis{\sphinxupquote{int}}\sphinxstyleliteralemphasis{\sphinxupquote{ (}}\sphinxstyleliteralemphasis{\sphinxupquote{optional}}\sphinxstyleliteralemphasis{\sphinxupquote{)}}) \textendash{} Expected Run Time before a drift is detected. Alibi detect uses this approach for it’s
online drift detectors. If the average ERT for the reference data is significantly higher
than the average run time for the drifted data, that might indicate a possible drift.

\item {} 
\sphinxAtStartPar
\sphinxstyleliteralstrong{\sphinxupquote{window\_size}} (\sphinxstyleliteralemphasis{\sphinxupquote{int}}) \textendash{} This parameter is used within Alibi’s online detectors.
It specifies the number of datapoints to include in one window.

\item {} 
\sphinxAtStartPar
\sphinxstyleliteralstrong{\sphinxupquote{n\_run}} (\sphinxstyleliteralemphasis{\sphinxupquote{int}}) \textendash{} This parameter is used within Alibi’s online detectors and specifies the number of runs
the detector must perform before we can get an average ERT.

\item {} 
\sphinxAtStartPar
\sphinxstyleliteralstrong{\sphinxupquote{n\_bootstraps}} (\sphinxstyleliteralemphasis{\sphinxupquote{int}}) \textendash{} This parameter is used within Alibi’s online detectors

\item {} 
\sphinxAtStartPar
\sphinxstyleliteralstrong{\sphinxupquote{context\_type}} (\sphinxstyleliteralemphasis{\sphinxupquote{str}}) \textendash{} Context that we wish to ignore
1) sub\sphinxhyphen{}population: if we wish to ignore the relative change in sub\sphinxhyphen{}population of certain
classes

\end{itemize}

\sphinxlineitem{Returns}
\sphinxAtStartPar
\begin{itemize}
\item {} 
\sphinxAtStartPar
\sphinxstyleemphasis{Lists and plots of relevant test statistics (p\sphinxhyphen{}values, distances) given the selected}

\item {} 
\sphinxAtStartPar
\sphinxstyleemphasis{detector (MMD, LSDD etc) and drift type (Sudden, Gradual, Online)}

\end{itemize}


\end{description}\end{quote}

\end{fulllineitems}

\index{run() (alibiDetectors.alibiDetectors method)@\spxentry{run()}\spxextra{alibiDetectors.alibiDetectors method}}

\begin{fulllineitems}
\phantomsection\label{\detokenize{alibiModules/alibiDetectors:alibiDetectors.alibiDetectors.run}}
\pysigstartsignatures
\pysiglinewithargsret{\sphinxbfcode{\sphinxupquote{run}}}{}{}
\pysigstopsignatures
\end{fulllineitems}


\end{fulllineitems}


\sphinxstepscope


\chapter{Feature Level Detectors}
\label{\detokenize{fldModules/modules:feature-level-detectors}}\label{\detokenize{fldModules/modules::doc}}
\sphinxstepscope


\section{baseModels module}
\label{\detokenize{fldModules/baseModels:module-baseModels}}\label{\detokenize{fldModules/baseModels:basemodels-module}}\label{\detokenize{fldModules/baseModels::doc}}\index{module@\spxentry{module}!baseModels@\spxentry{baseModels}}\index{baseModels@\spxentry{baseModels}!module@\spxentry{module}}\index{baseModels (class in baseModels)@\spxentry{baseModels}\spxextra{class in baseModels}}

\begin{fulllineitems}
\phantomsection\label{\detokenize{fldModules/baseModels:baseModels.baseModels}}
\pysigstartsignatures
\pysiglinewithargsret{\sphinxbfcode{\sphinxupquote{class\DUrole{w}{  }}}\sphinxcode{\sphinxupquote{baseModels.}}\sphinxbfcode{\sphinxupquote{baseModels}}}{\emph{\DUrole{n}{data}}, \emph{\DUrole{n}{sample\_size}}, \emph{\DUrole{n}{SBERT\_model}\DUrole{p}{:}\DUrole{w}{  }\DUrole{n}{Optional\DUrole{p}{{[}}str\DUrole{p}{{]}}}}}{}
\pysigstopsignatures
\sphinxAtStartPar
Bases: \sphinxcode{\sphinxupquote{object}}
\index{SBERT\_model (baseModels.baseModels attribute)@\spxentry{SBERT\_model}\spxextra{baseModels.baseModels attribute}}

\begin{fulllineitems}
\phantomsection\label{\detokenize{fldModules/baseModels:baseModels.baseModels.SBERT_model}}
\pysigstartsignatures
\pysigline{\sphinxbfcode{\sphinxupquote{SBERT\_model}}}
\pysigstopsignatures
\sphinxAtStartPar
This class sets the stage for the embedding models we choose to work with later
\begin{quote}\begin{description}
\sphinxlineitem{Return type}
\sphinxAtStartPar
An embedding model

\end{description}\end{quote}

\end{fulllineitems}

\index{doc2vec\_base() (baseModels.baseModels method)@\spxentry{doc2vec\_base()}\spxextra{baseModels.baseModels method}}

\begin{fulllineitems}
\phantomsection\label{\detokenize{fldModules/baseModels:baseModels.baseModels.doc2vec_base}}
\pysigstartsignatures
\pysiglinewithargsret{\sphinxbfcode{\sphinxupquote{doc2vec\_base}}}{\emph{\DUrole{n}{vector\_size}\DUrole{p}{:}\DUrole{w}{  }\DUrole{n}{Optional\DUrole{p}{{[}}int\DUrole{p}{{]}}}\DUrole{w}{  }\DUrole{o}{=}\DUrole{w}{  }\DUrole{default_value}{100}}, \emph{\DUrole{n}{window}\DUrole{p}{:}\DUrole{w}{  }\DUrole{n}{Optional\DUrole{p}{{[}}int\DUrole{p}{{]}}}\DUrole{w}{  }\DUrole{o}{=}\DUrole{w}{  }\DUrole{default_value}{2}}, \emph{\DUrole{n}{min\_count}\DUrole{p}{:}\DUrole{w}{  }\DUrole{n}{Optional\DUrole{p}{{[}}int\DUrole{p}{{]}}}\DUrole{w}{  }\DUrole{o}{=}\DUrole{w}{  }\DUrole{default_value}{1}}, \emph{\DUrole{n}{workers}\DUrole{p}{:}\DUrole{w}{  }\DUrole{n}{Optional\DUrole{p}{{[}}int\DUrole{p}{{]}}}\DUrole{w}{  }\DUrole{o}{=}\DUrole{w}{  }\DUrole{default_value}{4}}}{}
\pysigstopsignatures
\sphinxAtStartPar
Develops model for Doc2Vec embeddings

\end{fulllineitems}

\index{sbert\_base() (baseModels.baseModels method)@\spxentry{sbert\_base()}\spxextra{baseModels.baseModels method}}

\begin{fulllineitems}
\phantomsection\label{\detokenize{fldModules/baseModels:baseModels.baseModels.sbert_base}}
\pysigstartsignatures
\pysiglinewithargsret{\sphinxbfcode{\sphinxupquote{sbert\_base}}}{}{}
\pysigstopsignatures
\sphinxAtStartPar
Develops model for Sentence Transformer embeddings

\end{fulllineitems}


\end{fulllineitems}


\sphinxstepscope


\section{embedding module}
\label{\detokenize{fldModules/embedding:module-embedding}}\label{\detokenize{fldModules/embedding:embedding-module}}\label{\detokenize{fldModules/embedding::doc}}\index{module@\spxentry{module}!embedding@\spxentry{embedding}}\index{embedding@\spxentry{embedding}!module@\spxentry{module}}\index{embedding (class in embedding)@\spxentry{embedding}\spxextra{class in embedding}}

\begin{fulllineitems}
\phantomsection\label{\detokenize{fldModules/embedding:embedding.embedding}}
\pysigstartsignatures
\pysiglinewithargsret{\sphinxbfcode{\sphinxupquote{class\DUrole{w}{  }}}\sphinxcode{\sphinxupquote{embedding.}}\sphinxbfcode{\sphinxupquote{embedding}}}{\emph{\DUrole{o}{*}\DUrole{n}{args}}, \emph{\DUrole{o}{**}\DUrole{n}{kwargs}}}{}
\pysigstopsignatures
\sphinxAtStartPar
Bases: {\hyperref[\detokenize{baseModules/sampling:sampling.samplingData}]{\sphinxcrossref{\sphinxcode{\sphinxupquote{samplingData}}}}}, {\hyperref[\detokenize{baseModules/base:base.detectorParent}]{\sphinxcrossref{\sphinxcode{\sphinxupquote{detectorParent}}}}}
\index{\_\_init\_\_() (embedding.embedding method)@\spxentry{\_\_init\_\_()}\spxextra{embedding.embedding method}}

\begin{fulllineitems}
\phantomsection\label{\detokenize{fldModules/embedding:embedding.embedding.__init__}}
\pysigstartsignatures
\pysiglinewithargsret{\sphinxbfcode{\sphinxupquote{\_\_init\_\_}}}{\emph{\DUrole{o}{*}\DUrole{n}{args}}, \emph{\DUrole{o}{**}\DUrole{n}{kwargs}}}{}
\pysigstopsignatures
\sphinxAtStartPar
In this class, we turn the samples of text inputs into text embeddings, which we can then use
to a) either construct distributions, or b) calculate drift on. There are many different kinds
of text embeddings and encodings. In this class, we cover 3 umbrella embeddings (discussed below)
\begin{quote}\begin{description}
\sphinxlineitem{Return type}
\sphinxAtStartPar
A dictionary containing the embeddings as decided by the choice of embedding model and drift detection test type

\end{description}\end{quote}

\end{fulllineitems}

\index{dim\_reduction() (embedding.embedding method)@\spxentry{dim\_reduction()}\spxextra{embedding.embedding method}}

\begin{fulllineitems}
\phantomsection\label{\detokenize{fldModules/embedding:embedding.embedding.dim_reduction}}
\pysigstartsignatures
\pysiglinewithargsret{\sphinxbfcode{\sphinxupquote{dim\_reduction}}}{\emph{\DUrole{n}{emb\_dict}\DUrole{p}{:}\DUrole{w}{  }\DUrole{n}{Optional\DUrole{p}{{[}}dict\DUrole{p}{{]}}}\DUrole{w}{  }\DUrole{o}{=}\DUrole{w}{  }\DUrole{default_value}{None}}, \emph{\DUrole{n}{components}\DUrole{p}{:}\DUrole{w}{  }\DUrole{n}{Optional\DUrole{p}{{[}}int\DUrole{p}{{]}}}\DUrole{w}{  }\DUrole{o}{=}\DUrole{w}{  }\DUrole{default_value}{25}}, \emph{\DUrole{n}{n\_iters}\DUrole{p}{:}\DUrole{w}{  }\DUrole{n}{Optional\DUrole{p}{{[}}int\DUrole{p}{{]}}}\DUrole{w}{  }\DUrole{o}{=}\DUrole{w}{  }\DUrole{default_value}{7}}}{}
\pysigstopsignatures
\sphinxAtStartPar
Embeds text inherited from the sampling class.
\begin{quote}\begin{description}
\sphinxlineitem{Parameters}\begin{itemize}
\item {} 
\sphinxAtStartPar
\sphinxstyleliteralstrong{\sphinxupquote{emb\_dict}} (\sphinxstyleliteralemphasis{\sphinxupquote{dictionary}}) \textendash{} 

\item {} 
\sphinxAtStartPar
\sphinxstyleliteralstrong{\sphinxupquote{method}} (\sphinxstyleliteralemphasis{\sphinxupquote{Dictionary of embeddings as returned by the embed\_data}}) \textendash{} 

\item {} 
\sphinxAtStartPar
\sphinxstyleliteralstrong{\sphinxupquote{component}} (\sphinxstyleliteralemphasis{\sphinxupquote{int}}\sphinxstyleliteralemphasis{\sphinxupquote{ (}}\sphinxstyleliteralemphasis{\sphinxupquote{optional}}\sphinxstyleliteralemphasis{\sphinxupquote{)}}) \textendash{} 

\item {} 
\sphinxAtStartPar
\sphinxstyleliteralstrong{\sphinxupquote{SVD}} (\sphinxstyleliteralemphasis{\sphinxupquote{The number top components we want from PCA or}}) \textendash{} 

\item {} 
\sphinxAtStartPar
\sphinxstyleliteralstrong{\sphinxupquote{n\_iters}} (\sphinxstyleliteralemphasis{\sphinxupquote{int}}) \textendash{} 

\end{itemize}

\sphinxlineitem{Returns}
\sphinxAtStartPar
\begin{itemize}
\item {} 
\sphinxAtStartPar
\sphinxstyleemphasis{a dictionary containing the embeddings as decided by the choice of embedding model and}

\item {} 
\sphinxAtStartPar
\sphinxstyleemphasis{drift detection test type}

\end{itemize}


\end{description}\end{quote}

\end{fulllineitems}

\index{embed\_data() (embedding.embedding method)@\spxentry{embed\_data()}\spxextra{embedding.embedding method}}

\begin{fulllineitems}
\phantomsection\label{\detokenize{fldModules/embedding:embedding.embedding.embed_data}}
\pysigstartsignatures
\pysiglinewithargsret{\sphinxbfcode{\sphinxupquote{embed\_data}}}{}{}
\pysigstopsignatures
\sphinxAtStartPar
Embeds text inherited from the sampling class. The type of embedding (Doc2Vec, SBERT etc) is
decided by the user
\begin{quote}\begin{description}
\sphinxlineitem{Return type}
\sphinxAtStartPar
A dictionary containing the embeddings as decided by the choice of embedding model and drift detection test type

\end{description}\end{quote}

\end{fulllineitems}

\index{embed\_data\_iters() (embedding.embedding method)@\spxentry{embed\_data\_iters()}\spxextra{embedding.embedding method}}

\begin{fulllineitems}
\phantomsection\label{\detokenize{fldModules/embedding:embedding.embedding.embed_data_iters}}
\pysigstartsignatures
\pysiglinewithargsret{\sphinxbfcode{\sphinxupquote{embed\_data\_iters}}}{}{}
\pysigstopsignatures
\sphinxAtStartPar
Runs the embedding function “iterations” number of times, if the selected drift detection test
is the KS Test. For KL and JS Divergence, the iterations are taken care of in the distributions
class
\begin{quote}\begin{description}
\sphinxlineitem{Return type}
\sphinxAtStartPar
A dictionary containing the embeddings as decided by the choice of embedding model and drift detection test type

\end{description}\end{quote}

\end{fulllineitems}

\index{final\_embeddings() (embedding.embedding method)@\spxentry{final\_embeddings()}\spxextra{embedding.embedding method}}

\begin{fulllineitems}
\phantomsection\label{\detokenize{fldModules/embedding:embedding.embedding.final_embeddings}}
\pysigstartsignatures
\pysiglinewithargsret{\sphinxbfcode{\sphinxupquote{final\_embeddings}}}{}{}
\pysigstopsignatures\begin{quote}\begin{description}
\sphinxlineitem{Returns}
\sphinxAtStartPar
\begin{itemize}
\item {} 
\sphinxAtStartPar
\sphinxstyleemphasis{a dictionary containing the embeddings as decided by the choice of embedding model and}

\item {} 
\sphinxAtStartPar
\sphinxstyleemphasis{drift detection test type}

\end{itemize}


\end{description}\end{quote}

\end{fulllineitems}

\index{sampleData() (embedding.embedding method)@\spxentry{sampleData()}\spxextra{embedding.embedding method}}

\begin{fulllineitems}
\phantomsection\label{\detokenize{fldModules/embedding:embedding.embedding.sampleData}}
\pysigstartsignatures
\pysiglinewithargsret{\sphinxbfcode{\sphinxupquote{sampleData}}}{}{}
\pysigstopsignatures
\sphinxAtStartPar
Call the samplingData class to construct samples from the input data provided by the user
\begin{quote}\begin{description}
\sphinxlineitem{Return type}
\sphinxAtStartPar
Dictionary with samples for reference and comparison data (or streams of comparison data).

\end{description}\end{quote}

\end{fulllineitems}


\end{fulllineitems}


\sphinxstepscope


\section{distributions module}
\label{\detokenize{fldModules/distributions:module-distributions}}\label{\detokenize{fldModules/distributions:distributions-module}}\label{\detokenize{fldModules/distributions::doc}}\index{module@\spxentry{module}!distributions@\spxentry{distributions}}\index{distributions@\spxentry{distributions}!module@\spxentry{module}}\index{distributions (class in distributions)@\spxentry{distributions}\spxextra{class in distributions}}

\begin{fulllineitems}
\phantomsection\label{\detokenize{fldModules/distributions:distributions.distributions}}
\pysigstartsignatures
\pysiglinewithargsret{\sphinxbfcode{\sphinxupquote{class\DUrole{w}{  }}}\sphinxcode{\sphinxupquote{distributions.}}\sphinxbfcode{\sphinxupquote{distributions}}}{\emph{\DUrole{o}{*}\DUrole{n}{args}}, \emph{\DUrole{o}{**}\DUrole{n}{kwargs}}}{}
\pysigstopsignatures
\sphinxAtStartPar
Bases: {\hyperref[\detokenize{fldModules/embedding:embedding.embedding}]{\sphinxcrossref{\sphinxcode{\sphinxupquote{embedding}}}}}, {\hyperref[\detokenize{baseModules/sampling:sampling.samplingData}]{\sphinxcrossref{\sphinxcode{\sphinxupquote{samplingData}}}}}, {\hyperref[\detokenize{baseModules/base:base.detectorParent}]{\sphinxcrossref{\sphinxcode{\sphinxupquote{detectorParent}}}}}
\index{\_\_init\_\_() (distributions.distributions method)@\spxentry{\_\_init\_\_()}\spxextra{distributions.distributions method}}

\begin{fulllineitems}
\phantomsection\label{\detokenize{fldModules/distributions:distributions.distributions.__init__}}
\pysigstartsignatures
\pysiglinewithargsret{\sphinxbfcode{\sphinxupquote{\_\_init\_\_}}}{\emph{\DUrole{o}{*}\DUrole{n}{args}}, \emph{\DUrole{o}{**}\DUrole{n}{kwargs}}}{}
\pysigstopsignatures
\sphinxAtStartPar
In this class, we construct distributions out of the embeddings we got from the “embedding” class.
This is an optional class and is only required if we are running a distribution dependent test such
as KLD or JSD.
\begin{quote}\begin{description}
\sphinxlineitem{Returns}
\sphinxAtStartPar
\begin{itemize}
\item {} 
\sphinxAtStartPar
\sphinxstyleemphasis{A dictionary containing the distributions as decided by the choice of embedding model and}

\item {} 
\sphinxAtStartPar
\sphinxstyleemphasis{drift detection test type}

\end{itemize}


\end{description}\end{quote}

\end{fulllineitems}

\index{distributions\_doc2vec() (distributions.distributions method)@\spxentry{distributions\_doc2vec()}\spxextra{distributions.distributions method}}

\begin{fulllineitems}
\phantomsection\label{\detokenize{fldModules/distributions:distributions.distributions.distributions_doc2vec}}
\pysigstartsignatures
\pysiglinewithargsret{\sphinxbfcode{\sphinxupquote{distributions\_doc2vec}}}{}{}
\pysigstopsignatures
\sphinxAtStartPar
Constructs distributions for Doc2Vec embeddings
\begin{quote}\begin{description}
\sphinxlineitem{Return type}
\sphinxAtStartPar
A dictionary containing the distributions as decided by the choice of embedding model and drift detection test type

\end{description}\end{quote}

\end{fulllineitems}

\index{distributions\_seneconders() (distributions.distributions method)@\spxentry{distributions\_seneconders()}\spxextra{distributions.distributions method}}

\begin{fulllineitems}
\phantomsection\label{\detokenize{fldModules/distributions:distributions.distributions.distributions_seneconders}}
\pysigstartsignatures
\pysiglinewithargsret{\sphinxbfcode{\sphinxupquote{distributions\_seneconders}}}{}{}
\pysigstopsignatures
\sphinxAtStartPar
Constructs distributions for Sentence Transformer or Universal Sentence Encoder embeddings
\begin{quote}\begin{description}
\sphinxlineitem{Return type}
\sphinxAtStartPar
a dictionary containing the distributions as decided by the choice of embedding model and drift detector

\end{description}\end{quote}

\end{fulllineitems}

\index{final\_distributions() (distributions.distributions method)@\spxentry{final\_distributions()}\spxextra{distributions.distributions method}}

\begin{fulllineitems}
\phantomsection\label{\detokenize{fldModules/distributions:distributions.distributions.final_distributions}}
\pysigstartsignatures
\pysiglinewithargsret{\sphinxbfcode{\sphinxupquote{final\_distributions}}}{}{}
\pysigstopsignatures
\sphinxAtStartPar
Constructs distributions for the selected embeddings (Doc2Vec, SBERT, USE)
\begin{quote}\begin{description}
\sphinxlineitem{Return type}
\sphinxAtStartPar
A dictionary containing the distributions as decided by the choice of embedding model and drift detector

\end{description}\end{quote}

\end{fulllineitems}

\index{kde() (distributions.distributions method)@\spxentry{kde()}\spxextra{distributions.distributions method}}

\begin{fulllineitems}
\phantomsection\label{\detokenize{fldModules/distributions:distributions.distributions.kde}}
\pysigstartsignatures
\pysiglinewithargsret{\sphinxbfcode{\sphinxupquote{kde}}}{}{}
\pysigstopsignatures
\end{fulllineitems}


\end{fulllineitems}


\sphinxstepscope


\section{myDetectors module}
\label{\detokenize{fldModules/myDetectors:module-myDetectors}}\label{\detokenize{fldModules/myDetectors:mydetectors-module}}\label{\detokenize{fldModules/myDetectors::doc}}\index{module@\spxentry{module}!myDetectors@\spxentry{myDetectors}}\index{myDetectors@\spxentry{myDetectors}!module@\spxentry{module}}\index{myDetectors (class in myDetectors)@\spxentry{myDetectors}\spxextra{class in myDetectors}}

\begin{fulllineitems}
\phantomsection\label{\detokenize{fldModules/myDetectors:myDetectors.myDetectors}}
\pysigstartsignatures
\pysiglinewithargsret{\sphinxbfcode{\sphinxupquote{class\DUrole{w}{  }}}\sphinxcode{\sphinxupquote{myDetectors.}}\sphinxbfcode{\sphinxupquote{myDetectors}}}{\emph{\DUrole{o}{*}\DUrole{n}{args}}, \emph{\DUrole{o}{**}\DUrole{n}{kwargs}}}{}
\pysigstopsignatures
\sphinxAtStartPar
Bases: {\hyperref[\detokenize{fldModules/distributions:distributions.distributions}]{\sphinxcrossref{\sphinxcode{\sphinxupquote{distributions}}}}}, {\hyperref[\detokenize{fldModules/embedding:embedding.embedding}]{\sphinxcrossref{\sphinxcode{\sphinxupquote{embedding}}}}}, {\hyperref[\detokenize{baseModules/sampling:sampling.samplingData}]{\sphinxcrossref{\sphinxcode{\sphinxupquote{samplingData}}}}}, {\hyperref[\detokenize{baseModules/base:base.detectorParent}]{\sphinxcrossref{\sphinxcode{\sphinxupquote{detectorParent}}}}}
\index{\_\_init\_\_() (myDetectors.myDetectors method)@\spxentry{\_\_init\_\_()}\spxextra{myDetectors.myDetectors method}}

\begin{fulllineitems}
\phantomsection\label{\detokenize{fldModules/myDetectors:myDetectors.myDetectors.__init__}}
\pysigstartsignatures
\pysiglinewithargsret{\sphinxbfcode{\sphinxupquote{\_\_init\_\_}}}{\emph{\DUrole{o}{*}\DUrole{n}{args}}, \emph{\DUrole{o}{**}\DUrole{n}{kwargs}}}{}
\pysigstopsignatures
\sphinxAtStartPar
This class returns the final detection results based on the embeddings or distributions it
inherits. Currently, the tests covered in this class are Kolmogorov\textendash{}Smirnov test,
Kullback\textendash{}Leibler divergence, and Jensen\sphinxhyphen{}Shannon Divergence. Each test will return a different
output based on the kind of embedding model we choose to work with.
\begin{quote}\begin{description}
\sphinxlineitem{Parameters}\begin{itemize}
\item {} 
\sphinxAtStartPar
\sphinxstyleliteralstrong{\sphinxupquote{data\_ref}} (\sphinxstyleliteralemphasis{\sphinxupquote{np.ndarray}}\sphinxstyleliteralemphasis{\sphinxupquote{, }}\sphinxstyleliteralemphasis{\sphinxupquote{list}}) \textendash{} Dataset on which the original model is trained (ex: training dataset). We flag a drift
with a reference to the distribution of this dataset.

\item {} 
\sphinxAtStartPar
\sphinxstyleliteralstrong{\sphinxupquote{data\_h0}} (\sphinxstyleliteralemphasis{\sphinxupquote{np.ndarray}}\sphinxstyleliteralemphasis{\sphinxupquote{, }}\sphinxstyleliteralemphasis{\sphinxupquote{list}}) \textendash{} This is an optional dataset that we can use as a sanity check for the efficacy of a drift
detector. Generally, we use the same dataset as data\_ref (or a stream that comes soon after).
The lack of drift in data\_h0 (with data\_ref as our reference) is the necessary
condition to decide the robustness of the drift detection method

\item {} 
\sphinxAtStartPar
\sphinxstyleliteralstrong{\sphinxupquote{data\_h1}} (\sphinxstyleliteralemphasis{\sphinxupquote{np.ndarray}}\sphinxstyleliteralemphasis{\sphinxupquote{, }}\sphinxstyleliteralemphasis{\sphinxupquote{list}}) \textendash{} This is the principal dataset on which we might see a drift (ex. deployment data).
It can be just one sample (for sudden drifts) or stream of samples (for gradual drifts).
Often, for pipelines, datasets come in batches, and each new batch can then be updated
to the new data\_h1.

\item {} 
\sphinxAtStartPar
\sphinxstyleliteralstrong{\sphinxupquote{sample\_dict}} (\sphinxstyleliteralemphasis{\sphinxupquote{dict}}) \textendash{} Dictionary with samples for reference and comparison data (or streams of comparison data).
The user can directly input the dictionary as our dataset source if they would prefer to
organize the data on their own.

\item {} 
\sphinxAtStartPar
\sphinxstyleliteralstrong{\sphinxupquote{sample\_size}} (\sphinxstyleliteralemphasis{\sphinxupquote{int}}) \textendash{} This parameter decides the number of samples from each of the above 3 datasets that we would
like to work with. For instance, if the entire training data is 100K sentences, we can use
a sample\_size = 500 to randomly sample 500 of those sentences.

\item {} 
\sphinxAtStartPar
\sphinxstyleliteralstrong{\sphinxupquote{test}} (\sphinxstyleliteralemphasis{\sphinxupquote{str}}) \textendash{} Here, we specify the kind of drift detection test we want (KS, KLD, JSD, MMD, LSDD).
Each of them is described in greater detail in the README.md.

\item {} 
\sphinxAtStartPar
\sphinxstyleliteralstrong{\sphinxupquote{drift\_type}} (\sphinxstyleliteralemphasis{\sphinxupquote{str}}) \textendash{} Drifts can vary depending on the time horizan and frequency at which we try to detect
them. This parameter asks the user to specify the type of drift (“Sudden”, “Gradual”,
“Online”). The details of each are in README.md

\item {} 
\sphinxAtStartPar
\sphinxstyleliteralstrong{\sphinxupquote{plot}} (\sphinxstyleliteralemphasis{\sphinxupquote{bool}}) \textendash{} This parameter asks the user if they wish to see some of the plots of the results
from the drift detection. Not every detector will result in relevant plot.

\item {} 
\sphinxAtStartPar
\sphinxstyleliteralstrong{\sphinxupquote{windows}} (\sphinxstyleliteralemphasis{\sphinxupquote{int}}) \textendash{} This parameter is relevant for gradual drifts and helps break down the data into a
certain number of buckets. These buckets can act like “batches” or “data streams”.
The idea behind this approach is that we are trying to localize drifts to a certain
time frame and check for consistencies (or lack thereof) in detection.
If data\_h1 has 100K data points, and if we wish to detect drifts
gradually over time, a proxy approach would be to break the data in sets of 5K points
and then randomly sample from each set separately.

\item {} 
\sphinxAtStartPar
\sphinxstyleliteralstrong{\sphinxupquote{SBERT\_model}} (\sphinxstyleliteralemphasis{\sphinxupquote{str}}) \textendash{} This parameter is specific to the SBERT embedding models. If we choose to work with SBERT,
we can specify the type of SBERT embedding out here. Ex. ‘bert\sphinxhyphen{}base\sphinxhyphen{}uncased’

\item {} 
\sphinxAtStartPar
\sphinxstyleliteralstrong{\sphinxupquote{embedding\_model}} (\sphinxstyleliteralemphasis{\sphinxupquote{str}}) \textendash{} This is the principle parameter of this class. It decided the kind of embedding the text
goes through. The embeddings we consider thus far are:
a) SBERT: A Python framework for state\sphinxhyphen{}of\sphinxhyphen{}the\sphinxhyphen{}art sentence, text and image embeddings.
b) Universal Sentence Encoders: USE encodes text into high dimensional vectors that can be
used for text classification, semantic similarity, clustering, and other natural language tasks
c) Doc2Vec: a generalization of Word2Vec, which in turn is an algorithm that uses a
neural network model to learn word associations from a large corpus of text

\end{itemize}

\sphinxlineitem{Return type}
\sphinxAtStartPar
Drift detection related test statistics and any relevant plots

\end{description}\end{quote}

\end{fulllineitems}

\index{divergence\_doc2vec() (myDetectors.myDetectors method)@\spxentry{divergence\_doc2vec()}\spxextra{myDetectors.myDetectors method}}

\begin{fulllineitems}
\phantomsection\label{\detokenize{fldModules/myDetectors:myDetectors.myDetectors.divergence_doc2vec}}
\pysigstartsignatures
\pysiglinewithargsret{\sphinxbfcode{\sphinxupquote{divergence\_doc2vec}}}{}{}
\pysigstopsignatures
\sphinxAtStartPar
Calculated Kullback\textendash{}Leibler or Jensen\textendash{}Shannon Divergence for Doc2Vec embeddings
\begin{quote}\begin{description}
\sphinxlineitem{Return type}
\sphinxAtStartPar
The distances as given by the KL or JS Divergence

\end{description}\end{quote}

\end{fulllineitems}

\index{divergence\_seneconders() (myDetectors.myDetectors method)@\spxentry{divergence\_seneconders()}\spxextra{myDetectors.myDetectors method}}

\begin{fulllineitems}
\phantomsection\label{\detokenize{fldModules/myDetectors:myDetectors.myDetectors.divergence_seneconders}}
\pysigstartsignatures
\pysiglinewithargsret{\sphinxbfcode{\sphinxupquote{divergence\_seneconders}}}{}{}
\pysigstopsignatures
\sphinxAtStartPar
Calculated Kullback\textendash{}Leibler or Jensen\textendash{}Shannon Divergence for SBERT/USE embeddings
\begin{quote}\begin{description}
\sphinxlineitem{Return type}
\sphinxAtStartPar
The distances as given by the KL or JS Divergence

\end{description}\end{quote}

\end{fulllineitems}

\index{js\_divergence() (myDetectors.myDetectors method)@\spxentry{js\_divergence()}\spxextra{myDetectors.myDetectors method}}

\begin{fulllineitems}
\phantomsection\label{\detokenize{fldModules/myDetectors:myDetectors.myDetectors.js_divergence}}
\pysigstartsignatures
\pysiglinewithargsret{\sphinxbfcode{\sphinxupquote{js\_divergence}}}{\emph{\DUrole{n}{p}}, \emph{\DUrole{n}{q}}}{}
\pysigstopsignatures
\sphinxAtStartPar
Calculated the Jensen\textendash{}Shannon Divergence for the 2 distributions p and q
\begin{quote}\begin{description}
\sphinxlineitem{Parameters}\begin{itemize}
\item {} 
\sphinxAtStartPar
\sphinxstyleliteralstrong{\sphinxupquote{p}} (\sphinxstyleliteralemphasis{\sphinxupquote{np.ndarray}}) \textendash{} 

\item {} 
\sphinxAtStartPar
\sphinxstyleliteralstrong{\sphinxupquote{data}} (\sphinxstyleliteralemphasis{\sphinxupquote{A numpy array containing the distributions of some}}) \textendash{} 

\item {} 
\sphinxAtStartPar
\sphinxstyleliteralstrong{\sphinxupquote{q}} (\sphinxstyleliteralemphasis{\sphinxupquote{np.ndarray}}) \textendash{} 

\item {} 
\sphinxAtStartPar
\sphinxstyleliteralstrong{\sphinxupquote{data}} \textendash{} 

\end{itemize}

\sphinxlineitem{Return type}
\sphinxAtStartPar
The JS Divergence distance

\end{description}\end{quote}

\end{fulllineitems}

\index{kl\_divergence() (myDetectors.myDetectors method)@\spxentry{kl\_divergence()}\spxextra{myDetectors.myDetectors method}}

\begin{fulllineitems}
\phantomsection\label{\detokenize{fldModules/myDetectors:myDetectors.myDetectors.kl_divergence}}
\pysigstartsignatures
\pysiglinewithargsret{\sphinxbfcode{\sphinxupquote{kl\_divergence}}}{\emph{\DUrole{n}{p}}, \emph{\DUrole{n}{q}}}{}
\pysigstopsignatures
\sphinxAtStartPar
Calculated the Kullback\textendash{}Leibler Divergence for the 2 distributions p and q
\begin{quote}\begin{description}
\sphinxlineitem{Parameters}\begin{itemize}
\item {} 
\sphinxAtStartPar
\sphinxstyleliteralstrong{\sphinxupquote{p}} (\sphinxstyleliteralemphasis{\sphinxupquote{np.ndarray}}) \textendash{} 

\item {} 
\sphinxAtStartPar
\sphinxstyleliteralstrong{\sphinxupquote{data}} (\sphinxstyleliteralemphasis{\sphinxupquote{A numpy array containing the distributions of some}}) \textendash{} 

\item {} 
\sphinxAtStartPar
\sphinxstyleliteralstrong{\sphinxupquote{q}} (\sphinxstyleliteralemphasis{\sphinxupquote{np.ndarray}}) \textendash{} 

\item {} 
\sphinxAtStartPar
\sphinxstyleliteralstrong{\sphinxupquote{data}} \textendash{} 

\end{itemize}

\sphinxlineitem{Return type}
\sphinxAtStartPar
The KL Divergence distance

\end{description}\end{quote}

\end{fulllineitems}

\index{ks\_doc2vec() (myDetectors.myDetectors method)@\spxentry{ks\_doc2vec()}\spxextra{myDetectors.myDetectors method}}

\begin{fulllineitems}
\phantomsection\label{\detokenize{fldModules/myDetectors:myDetectors.myDetectors.ks_doc2vec}}
\pysigstartsignatures
\pysiglinewithargsret{\sphinxbfcode{\sphinxupquote{ks\_doc2vec}}}{}{}
\pysigstopsignatures
\sphinxAtStartPar
Calculated Kolmogorov\textendash{}Smirnov test for Doc2Vec embeddings
\begin{quote}\begin{description}
\sphinxlineitem{Return type}
\sphinxAtStartPar
The p\sphinxhyphen{}values and distances as given by the Kolmogorov\textendash{}Smirnov test

\end{description}\end{quote}

\end{fulllineitems}

\index{ks\_sbert() (myDetectors.myDetectors method)@\spxentry{ks\_sbert()}\spxextra{myDetectors.myDetectors method}}

\begin{fulllineitems}
\phantomsection\label{\detokenize{fldModules/myDetectors:myDetectors.myDetectors.ks_sbert}}
\pysigstartsignatures
\pysiglinewithargsret{\sphinxbfcode{\sphinxupquote{ks\_sbert}}}{}{}
\pysigstopsignatures
\sphinxAtStartPar
Calculated the Kolmogorov\textendash{}Smirnov test test for SBERT embeddings.
\begin{quote}\begin{description}
\sphinxlineitem{Return type}
\sphinxAtStartPar
The p\sphinxhyphen{}values and distances as given by the Kolmogorov\textendash{}Smirnov test

\end{description}\end{quote}

\end{fulllineitems}

\index{run() (myDetectors.myDetectors method)@\spxentry{run()}\spxextra{myDetectors.myDetectors method}}

\begin{fulllineitems}
\phantomsection\label{\detokenize{fldModules/myDetectors:myDetectors.myDetectors.run}}
\pysigstartsignatures
\pysiglinewithargsret{\sphinxbfcode{\sphinxupquote{run}}}{}{}
\pysigstopsignatures
\sphinxAtStartPar
Calculates the drift detection metrics, as specified by the choice of embedding model and
drift detection test.
\begin{quote}\begin{description}
\sphinxlineitem{Return type}
\sphinxAtStartPar
Distances for KLD or JSD or P\sphinxhyphen{}values for KS (depending on choice of test)

\end{description}\end{quote}

\end{fulllineitems}


\end{fulllineitems}



\chapter{Indices and tables}
\label{\detokenize{index:indices-and-tables}}\begin{itemize}
\item {} 
\sphinxAtStartPar
\DUrole{xref,std,std-ref}{genindex}

\item {} 
\sphinxAtStartPar
\DUrole{xref,std,std-ref}{modindex}

\item {} 
\sphinxAtStartPar
\DUrole{xref,std,std-ref}{search}

\end{itemize}


\renewcommand{\indexname}{Python Module Index}
\begin{sphinxtheindex}
\let\bigletter\sphinxstyleindexlettergroup
\bigletter{a}
\item\relax\sphinxstyleindexentry{alibiDetectors}\sphinxstyleindexpageref{alibiModules/alibiDetectors:\detokenize{module-alibiDetectors}}
\indexspace
\bigletter{b}
\item\relax\sphinxstyleindexentry{base}\sphinxstyleindexpageref{baseModules/base:\detokenize{module-base}}
\item\relax\sphinxstyleindexentry{baseModels}\sphinxstyleindexpageref{fldModules/baseModels:\detokenize{module-baseModels}}
\item\relax\sphinxstyleindexentry{basicDetectors}\sphinxstyleindexpageref{alibiModules/basicDetectors:\detokenize{module-basicDetectors}}
\indexspace
\bigletter{d}
\item\relax\sphinxstyleindexentry{distributions}\sphinxstyleindexpageref{fldModules/distributions:\detokenize{module-distributions}}
\indexspace
\bigletter{e}
\item\relax\sphinxstyleindexentry{embedding}\sphinxstyleindexpageref{fldModules/embedding:\detokenize{module-embedding}}
\indexspace
\bigletter{m}
\item\relax\sphinxstyleindexentry{myDetectors}\sphinxstyleindexpageref{fldModules/myDetectors:\detokenize{module-myDetectors}}
\indexspace
\bigletter{o}
\item\relax\sphinxstyleindexentry{onlineDetectors}\sphinxstyleindexpageref{alibiModules/onlineDetectors:\detokenize{module-onlineDetectors}}
\indexspace
\bigletter{s}
\item\relax\sphinxstyleindexentry{sampling}\sphinxstyleindexpageref{baseModules/sampling:\detokenize{module-sampling}}
\end{sphinxtheindex}

\renewcommand{\indexname}{Index}
\printindex
\end{document}